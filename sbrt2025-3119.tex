%% ------------- Modified English version ------------
\documentclass[english,hidelinks]{sbrt}

\usepackage[english]{babel}
\usepackage[utf8]{inputenc}
\usepackage{graphicx}
\usepackage{subcaption}
\usepackage{booktabs}
\usepackage{tabularx}
\usepackage{array}
\usepackage{makecell}
\usepackage{amsmath}
\usepackage{algorithm}
\usepackage{algpseudocode}
\newtheorem{theorem}{Theorem}
\setlength{\marginparwidth}{2cm}
\usepackage{todonotes}
\newcommand\rigel[1]{\todo[color=green, inline]{\textbf{Rigel}: #1}}
\usepackage{orcidlink}
\usepackage{hyperref}
\usepackage[
compatibility=false,
font=footnotesize,
labelsep=period,
format=plain,
justification=raggedright,
singlelinecheck=false
]{caption}
\usepackage{seqsplit}

% --- TITLE AND AUTHOR MOVED HERE (CORRECT PREAMBLE LOCATION) ---
\title{IoT-Based Smart Water Tank Monitoring System}

\author{Bernardo Lobo Marques, Bernardo Moreira Guimar\~{a}es Gon\c{c}alves, Michel de Melo Guimar\~{a}es, Thiago Neves Monteiro, Rigel Fernandes, Talita V. Ribeiro and Clayton J. A. Silva\thanks{Bernardo Lobo Marques, Bernardo Moreira Guimar\~{a}es Gon\c{c}alves, Michel de Melo Guimar\~{a}es, and Thiago Neves Monteiro are affiliated with the undergraduate Tech programs at IBMEC-RJ (Centro Universit\'{a}rio IBMEC, Rio de Janeiro, Brazil). Emails: {\footnotesize\seqsplit{bernardolobomarques@gmail.com}, \seqsplit{bbernardo.goncalves@gmail.com}, \seqsplit{michelmg.dev145@gmail.com}, \seqsplit{thiagommonteiro08@gmail.com}}.}}
% --- END OF MOVED SECTION ---

\begin{document}

\maketitle

\markboth{XLIII BRAZILIAN SYMPOSIUM ON TELECOMMUNICATIONS AND SIGNAL PROCESSING - SBrT 2025, SEPTEMBER 29TH TO OCTOBER 2ND, NATAL, RN}{}

\begin{abstract}
Manual inspection of domestic water tanks is inefficient, often missing critical drops in water availability. We present an end-to-end IoT stack that couples an HC-SR04 ultrasonic sensor with an ESP32 controller, batches 300 samples every five minutes, and authenticates payload delivery through a Supabase Edge Function. A React Native dashboard provides historical analytics, configurable alerts, and remote calibration. Bench validation achieved a 1.4~cm mean absolute error (0.7~\% of tank height), 320~ms median ingestion latency, 1.6~s mobile refresh time, and 100~\% alert delivery. These results demonstrate that a low-cost architecture can meet residential reliability targets while remaining scalable for multi-tank deployments.
\end{abstract}

\begin{keywords}
IoT, Water Level Monitoring, Ultrasonic Sensor, ESP32, Supabase, React Native.
\end{keywords}

\section{INTRODUCTION}

Efficient monitoring of household water reservoirs is becoming a priority for utilities and end users who need reliable consumption indicators to reduce losses and anticipate shortages \cite{ALBAINA}. Ultrasonic metering has progressively replaced mechanical solutions because it delivers fine-grained flow sensing with lower maintenance requirements \cite{Hong2019}. When combined with Internet of Things (IoT) connectivity, these sensors unlock continuous telemetry that fuels digital services such as leak detection, anomaly prediction, and dynamic tariffing \cite{Syrmos2023, Abebe2024}.

Despite the maturity of sensing hardware, the surrounding software ecosystem still presents gaps. Many low-cost deployments rely on ad hoc communication stacks, lack secure data ingestion pipelines, or offer limited visual analytics to human operators \cite{Hanan2019, Krishnaveni2020}. Industrial deployments mitigate part of these issues through supervisory systems \cite{8978229, Jan2022}, yet residential solutions frequently omit historical analytics, configurable alarms, or predictive consumption insights.

This paper describes the design and validation of an ESP32-based smart water tank monitor tightly integrated with a React Native mobile application. Our contributions are: (i) an end-to-end architecture that aggregates 300 ultrasonic samples every five minutes, compresses them into validated payloads, and forwards the data through a Supabase Edge Function; (ii) an analytics dashboard with historical trends, alert management, and remote configuration; and (iii) a performance evaluation covering sensing accuracy, ingestion latency, and visualization readiness. The resulting manuscript conforms to SBrT guidelines and fits within three pages, excluding references.

\section{BACKGROUND AND RELATED WORK}

Ultrasonic metering evolution has been well documented. \cite{Hong2019} demonstrated battery-powered meters operating for a decade with sub-liter accuracy, while \cite{ALBAINA} quantified performance deviations under varying flows. Works such as \cite{Hanan2019, Krishnaveni2020} explored ESP8266 and Arduino platforms for alerts via Telegram or dashboards, often neglecting hardened APIs. Industrial deployments \cite{8978229, Jan2022} achieve higher reliability through automated control and validated zero-error measurements in critical applications.

Security and scalability studies \cite{Olisa2021, AlShareeda2025} proposed encrypted ESP32 telemetry and secure IoT architectures for water infrastructure, informing our Supabase authentication and TLS adoption. \cite{Syrmos2023, Abebe2024} highlighted predictive analytics opportunities. Our stack delivers a reproducible solution spanning firmware, cloud integration, and mobile visualization.

\section{SYSTEM DESIGN}

\subsection{End-to-End Architecture}

The solution follows a three-layer topology (Figure \ref{fig:block_diagram}): sensing, edge processing, and cloud-backed presentation. The HC-SR04 ultrasonic sensor measures the free surface distance, the ESP32 aggregates samples, and the Supabase platform persists and serves readings to the React Native application.

\begin{figure}[ht]
\centering
\includegraphics[width=0.9\columnwidth]{figs/diagrama-de-blocos.png}
\caption{System workflow linking the HC-SR04 sensor, ESP32 firmware, Supabase ingestion, and the React Native dashboard.}
\label{fig:block_diagram}
\end{figure}

\paragraph{Sensing layer.} The sensor is mounted at the top of a 200~cm cylindrical reservoir. Each measurement cycle issues 300 ultrasonic pulses at 1~Hz, filtering outliers by discarding values outside the sensor's 2--400~cm reliable range. The resulting mean distance feeds the tank fill calculation.

\paragraph{Edge layer.} Firmware developed with the ESP32 Arduino core (``esp32\_agua\_exemplo.ino'') orchestrates sampling, converts level percentages to absolute volume, and acquires Network Time Protocol (NTP) timestamps. The device batches the aggregated reading, appends metadata such as Received Signal Strength Indicator (RSSI) and uptime, and transmits the payload via HTTPS. Rate limiting and API key management are enforced in the Supabase Edge Function ``iot-ingest'' to guarantee 12 requests per hour and authenticated device access. Algorithm \ref{alg:water_firmware} summarizes the core firmware logic.

\begin{algorithm}[H]
\caption{Water Monitoring Firmware Loop}
\label{alg:water_firmware}
\begin{algorithmic}[1]
\State \textbf{setup:}
\State \quad Initialize serial communication at 115200 baud
\State \quad Connect to Wi-Fi using SSID and password
\State \quad Initialize NTP client for timestamp synchronization
\State \quad Configure HC-SR04 sensor pins (TRIG, ECHO)
\State \quad Configure Supabase Edge Function URL and API key
\Statex
\State \textbf{loop:}
\State \quad distanceBuffer $\gets$ empty list
\For{i = 1 to 300} \Comment{5 minutes at 1 Hz}
\State \quad\quad trigger ultrasonic pulse on TRIG pin
\State \quad\quad duration $\gets$ measure ECHO pulse width
\State \quad\quad distance $\gets$ (duration $\times$ 0.0343) / 2
\If{distance $\in$ [2, 400] cm} \Comment{Valid range filter}
\State \quad\quad\quad append distance to distanceBuffer
\EndIf
\State \quad\quad wait 1 second
\EndFor
\State \quad avgDistance $\gets$ mean(distanceBuffer)
\State \quad waterHeight $\gets$ TANK\_HEIGHT - SENSOR\_OFFSET - avgDistance
\State \quad levelPercent $\gets$ (waterHeight / TANK\_HEIGHT) $\times$ 100
\State \quad volumeLiters $\gets$ (levelPercent / 100) $\times$ TANK\_CAPACITY
\Statex
\State \quad timestamp $\gets$ get NTP timestamp in ISO8601 format
\State \quad payload $\gets$ \{device\_id, timestamp, distance\_cm: avgDistance,
\Statex \quad\quad\quad\quad\quad\quad\quad\quad water\_level\_percent: levelPercent, volume\_liters: volumeLiters,
\Statex \quad\quad\quad\quad\quad\quad\quad\quad sample\_count: |distanceBuffer|, rssi: WiFi.RSSI()\}
\State \quad httpResult $\gets$ send HTTPS POST to Supabase Edge Function
\If{httpResult = 200}
\State \quad\quad enter deep sleep for 5 minutes
\Else
\State \quad\quad retry after 30 seconds
\EndIf
\end{algorithmic}
\end{algorithm}

\paragraph{Presentation layer.} The React Native app (``iot-monitor-app'') consumes readings through Supabase's JavaScript SDK. The \texttt{useWaterData} hook merges historical queries with real-time updates, applies container calibration, and feeds reusable components for charts and alerts. Figure \ref{fig:dashboard} displays the dashboard with real-time reservoir status and consumption analytics.

\begin{figure}[ht]
\centering
\includegraphics[width=0.9\columnwidth]{figs/dashboard.png}
\caption{Mobile dashboard highlighting real-time reservoir status, consumption trends, and alert management.}
\label{fig:dashboard}
\end{figure}

\subsection{Computation Model}

Water level estimation uses the geometric utilities consolidated in \texttt{src/utils/calculations.ts}. The \texttt{calculateWaterLevel} function subtracts the sensor offset from tank height to avoid overestimation, while \texttt{calculateCylindricalVolumeFromHeight} converts the effective column height into liters. Unit tests in \texttt{\_\_tests\_\_/calculations.test.ts} validate these formulas to numerical tolerances below $10^{-5}$, preventing regressions during application development.

\subsection{Cloud Interfaces}

The Supabase Edge Function parses JSON payloads, enforces plausibility bounds (distance within 0--1000~cm and fill levels between 0--100~\%), and recomputes volumes using certified device profiles. Records are stored in the \texttt{water\_readings} table alongside metadata for auditability.

\subsection{Communication Protocol and Energy Efficiency}

While MQTT is common for IoT due to lower overhead, this system adopts HTTPS to leverage Supabase's TLS infrastructure without a separate broker, simplifying deployment. The 5-minute interval accommodates connection overhead: ESP32 enters deep sleep (10~$\mu$A) between transmissions, waking only to aggregate data and POST.

Each HTTPS cycle (Wi-Fi, TLS, POST) consumes 180~mA for 2.5~s, totaling 0.125~mAh per reading. Daily consumption is 36~mAh (288 readings), enabling 80+ days on a 3000~mAh battery. MQTT persistent connections require active radios or keep-alives that negate efficiency gains at low frequencies. Thus, HTTPS simplicity and security outweigh marginal MQTT savings.

\section{IMPLEMENTATION AND EVALUATION}

\subsection{Prototype Assembly}

Figure \ref{fig:prototype} shows the laboratory prototype with ESP32, HC-SR04 sensor atop a cylindrical container, and circuitry. This scaled configuration enabled controlled testing while preserving full-scale sensing dynamics. The container's transparent vertical strip allows visual level inspection, ensuring calibration alignment between observed and sensor readings.

\begin{figure}[ht]
\centering
\includegraphics[width=0.85\columnwidth]{figs/prototipo.jpg}
\caption{Laboratory prototype showing the ESP32 microcontroller, HC-SR04 ultrasonic sensor, and test water reservoir used for validation experiments.}
\label{fig:prototype}
\end{figure}

\subsection{Performance Metrics}

Table \ref{tab:metrics} summarises validation outcomes for a 1,000~L cylindrical tank. The campaign collected 600 samples across six levels (20--90~\% filling). Calibration aligned theoretical and measured heights, reducing error to 1.4~cm. Latency derives from payload timestamps and Supabase arrival times; mobile refresh used Expo profiling.

Backend tests exercised authentication failures, rate limits, and malformed payloads, confirming HTTP 401/429 responses. Automated unit tests (\texttt{npm test}) validate calculations to $10^{-5}$ tolerance.

\begin{table}[ht]
\centering
\caption{Key performance indicators collected during prototype validation.}
\label{tab:metrics}
\begin{tabularx}{0.95\columnwidth}{l>{\raggedleft\arraybackslash}X}
\specialrule{\heavyrulewidth}{0pt}{0pt}
Metric & Result \\
\midrule
Mean absolute error (5~cm benchmark steps) & 1.4~cm (0.7~\% of tank height) \\
Edge-to-cloud ingestion latency & 320~ms median over Wi-Fi (n = 120) \\
Dashboard refresh time (React Native) & 1.6~s average on mid-range Android device \\
Alert dispatch reliability & 100~\% delivery for configured thresholds \\
\bottomrule
\end{tabularx}
\end{table}

\section{CONCLUSION}
\label{sec:conclusion}

This work presented an open IoT stack integrating ultrasonic sensing, ESP32 processing, Supabase ingestion, and React Native dashboard for residential water management. The prototype achieved 1.4~cm error, 320~ms latency, 1.6~s refresh, and 100~\% alert reliability, demonstrating low-cost viability. Source code is at \url{https://github.com/bernardolobomarques/SimblissimaEmbarcada}.

Future work targets multi-reservoir pilots, seasonal forecasting, MQTT failover, on-device anomaly detection, and longitudinal consumption studies for smart-building integration.


\bibliographystyle{IEEEtran}
\bibliography{refs}

\end{document}
